\documentclass[8pt]{article}

\usepackage{extsizes}

\usepackage[utf8]{inputenc}
\usepackage[T1]{fontenc}
\usepackage[top=15mm,bottom=25mm,left=25mm,right=25mm]{geometry}
\usepackage[ngerman]{babel}
% \usepackage{hyphenat}
\usepackage{bookman}
% \usepackage{times}
\usepackage{fontawesome}
\usepackage{multicol}
\usepackage{tikz}
\usepackage[skins]{tcolorbox}
\usepackage{fancyhdr}
% \usepackage[dvipsnames]{xcolor}
\usepackage{enumitem}

\definecolor{cvColor}{rgb}{0.25, 0.25, 0.5}
\definecolor{cvBackgroundColor}{rgb}{0.94, 0.97, 0.99}
\definecolor{cvFooterColor}{rgb}{0.4, 0.4, 0.4}
\definecolor{cvContentColor}{rgb}{0.3, 0.3, 0.3}
\definecolor{cvSubsectionColor}{rgb}{0.15, 0.15, 0.15}

\newcommand{\cvFirstName}{Markus}
\newcommand{\cvLastName}{Pawellek}
\newcommand{\cvAddress}{Arvid-Harnack-Straße 12 \\ 07743 Jena \\ Deutschland}
\newcommand{\cvBirthday}{7.~Mai~1995}
\newcommand{\cvBirthplace}{Meiningen,~Deutschland}
\newcommand{\cvMobile}{+49~173~7262913}
\newcommand{\cvMail}{markuspawellek@gmail.com}
\newcommand{\cvGitHub}{lyrahgames}
\newcommand{\cvPhoto}{photo.png}

\newcommand{\cvCV}{Curriculum~Vitae}
\newcommand{\cvName}{\cvFirstName~\cvLastName}
\newcommand{\cvFooterStyle}{%
  \normalfont%
  \sffamily%
  \bfseries%
  \color{cvFooterColor}%
  \footnotesize%
  \scshape%
}
\newcommand{\cvFooterSeparator}{\quad $\cdot$ \quad}
\newcommand{\cvHeadCVStyle}{%
  \normalfont%
  \footnotesize%
  \itshape%
  \color{cvColor}%
}
\newcommand{\cvHeadCVRule}{%
  \parbox{0.3\linewidth}{\rule{\linewidth}{0.5pt}}%
}
\newcommand{\cvHeadNameBaseStyle}{%
  \normalfont%
  \sffamily%
  \Huge%
}
\newcommand{\cvHeadLastNameStyle}{%
  \color{cvColor}%
  \bfseries%
}
\newcommand{\cvHeadAddressStyle}{%
  \normalfont%
  \small%
  \sffamily%
  % \color{cvContentColor}
}
\newcommand{\cvHeadContactStyle}{%
  \normalfont%
  \small%
  \sffamily%
}
\newcommand{\cvHeadBirthStyle}{%
  \normalfont%
  \small%
  \sffamily%
  \color{cvContentColor}%
  \itshape%
  % \scshape%
  % \bfseries%
}
\newcommand{\cvSectionStyle}{%
  \normalfont%
  \Large%
  \color{cvColor}%
  \bfseries%
  \sffamily%
}
\newcommand{\cvSubsectionStyle}{%
  \normalfont%
  \sffamily%
  \itshape%
  \bfseries%
  \color{cvSubsectionColor}%
}
\newcommand{\cvTimeStyle}{%
  \normalfont%
  \sffamily%
  \footnotesize%
  \itshape%
}

\fancypagestyle{cvPagestyle}{
  \fancyhf{}
  \renewcommand{\headrulewidth}{0pt}
  \renewcommand{\footrulewidth}{0pt}
  \fancyfoot[L]{%
    \bigskip%
    \cvFooterStyle%
    \today
  }
  \fancyfoot[R]{%
    \bigskip%
    \cvFooterStyle%
    \cvName%
    \cvFooterSeparator%
    \cvCV%
  }
}

\linespread{1.2}
\pagestyle{cvPagestyle}

\newcommand{\cvQuote}[1]{\guillemotright{}#1\guillemotleft}

\newcommand{\cvHead}{
  \begin{minipage}[c]{0.7\linewidth}
    \begin{center}
      {%
        \cvHeadNameBaseStyle%
        \cvFirstName~\cvHeadLastNameStyle\cvLastName%
      } \\[0.2em]
      {%
        \cvHeadCVStyle%
        \cvHeadCVRule~~\cvCV~~\cvHeadCVRule%
      } \\[1em]
      \begin{minipage}[c]{0.46\linewidth}
        \raggedleft%
        \cvHeadAddressStyle%
        \cvAddress%
      \end{minipage}
      \hfill%
      \vrule%
      \hfill%
      \begin{minipage}[c]{0.46\linewidth}
        \newcommand{\iconBox}[1]{\parbox{1.5em}{\centering ##1}}%
        \cvHeadContactStyle%
        \iconBox{\faMobile} \cvMobile \\
        \iconBox{\faEnvelopeO} \cvMail \\
        \iconBox{\faGithub} \cvGitHub
      \end{minipage}\\[2.5em]
      {%
        \cvHeadBirthStyle%
        Geboren am \cvBirthday{} in \cvBirthplace%
      }%
    \end{center}
  \end{minipage}
  \hfill%
  \begin{minipage}[c]{0.28\linewidth}
    \begin{tikzpicture}
      \node[circle,draw=black,line width=1pt, inner sep=0.25\linewidth, fill overzoom image=\cvPhoto] () {};
    \end{tikzpicture}
  \end{minipage}
  \bigskip%
}

\newcommand{\cvSection}[1]{%
  \smallskip%
  {%
    \cvSectionStyle #1%
  }\\[-0.5em]
  \rule{\linewidth}{0.8pt}%
  \par%
  \smallskip%
}

\newcommand{\cvSubsection}[1]{%
  \begin{tcolorbox}[left=0pt, top=0pt, bottom=0pt, right=0pt, boxsep=5pt, arc=5pt, frame code={}, colback=cvBackgroundColor]
    \cvSubsectionStyle #1%
  \end{tcolorbox}
}

\newenvironment{cvItemize}{%
  \begin{itemize}[itemsep=0mm, leftmargin=4mm]
}{%
  \end{itemize}
}

\newenvironment{cvTimeItem}[2]{
  \par
  \begin{minipage}[c]{0.15\linewidth}
    \raggedleft
    \cvTimeStyle #1
  \end{minipage}
  \quad
  \vrule
  \quad
  \begin{minipage}[t]{0.79\linewidth}
    \sffamily\textsc{\color{cvColor} \textbf{#2}}\par
    \normalfont\footnotesize\sffamily\color{cvContentColor}
}{
  \end{minipage}
  \par%
  \vspace{\baselineskip}%
}

\newenvironment{cvSkillItem}[2]{
  \par
  \begin{minipage}[c]{0.2\linewidth}
    \raggedleft
    \normalfont
    \sffamily
    % \small
    \itshape
    #1
  \end{minipage}
  \hspace{0.02\linewidth}
  \vrule
  \hspace{0.02\linewidth}
  \begin{minipage}[t]{0.74\linewidth}
    \sffamily\textsc{\color{cvColor} \textbf{#2}}\par
    \normalfont\footnotesize\sffamily\color{cvContentColor}
}{
  \end{minipage}
  \par%
  \vspace{\baselineskip}%
}

\setlength{\parindent}{0mm}

\begin{document}
  \cvHead

  \cvSection{Ausbildung}
    \cvSubsection{Goetheschule Ilmenau Staatliches Gymnasium}
      \begin{cvTimeItem}{Sep.~2009 - Jun.~2013}{Allgemeine Hochschulreife (1,2)}
        \begin{cvItemize}
          \item Besuch der mathematisch-naturwissenschaftlichen Spezialklassen
          \item Abschluss mit sehr gutem Erfolg in den erweiterten Fächern Mathematik, Physik und Informatik
          \item Abgabe zweier Facharbeiten in den Bereichen \cvQuote{Compilerbau} und \cvQuote{Raytracing}
          \item Preisträger mehrerer Mathematik- und Physikolympiaden
          \item Dreijähriger Besuch der Elektronik-AG
        \end{cvItemize}
      \end{cvTimeItem}

    \cvSubsection{Technische Universität Ilmenau}
      \begin{cvTimeItem}{Okt. 2011 - Sep. 2012}{Frühstudium: Experimentalphysik (1,0)}
      \end{cvTimeItem}

    \cvSubsection{Friedrich-Schiller-Universität Jena}
      \begin{cvTimeItem}{Okt. 2013 - Sep. 2017}{B.Sc. Physik (1,7)}
        Abschlussarbeit \cvQuote{Generierung von Irradiance Maps} (1,3) über das Cachen der diffusen Lichtverteilung einer Szene, um deren Darstellung in Echtzeit mithilfe des Raytracing-Algorithmus zu ermöglichen
      \end{cvTimeItem}
      \begin{cvTimeItem}{Okt. 2015 - Sep. 2018}{B.Sc. Mathematik (1,4)}
        \begin{cvItemize}
          \item Spezialisierung auf den Bereich \cvQuote{Theoretische Informatik}
          \item Abschlussarbeit \cvQuote{Implementierung einer Finite-Elemente-Methode auf der Grafikkarte} (1,0) über die numerische Simulation von Lösungen der idealen Wellengleichung auf zweidimensionalen Untermannigfaltigkeiten
        \end{cvItemize}
      \end{cvTimeItem}
      \begin{cvTimeItem}{seit Okt.~2017}{M.Sc. Physik}
        \begin{cvItemize}
          \item Spezialisierung auf den Bereich \cvQuote{Quanten- und Gravitationstheorie} mit Nebenfach \cvQuote{Astronomie}
          \item Abschlussarbeit \cvQuote{Design and Implementation of Pseudo Random Number Generators for Simulation in Physics}
        \end{cvItemize}
      \end{cvTimeItem}

  \cvSection{Fähigkeiten}
    \cvSubsection{Sprachen}
      \vspace{-1.8em}
      \begin{multicols}{2}
        \begin{cvSkillItem}{Deutsch}{Muttersprache}
        \end{cvSkillItem}
        \begin{cvSkillItem}{Englisch}{Fließend in Wort und Schrift}
          Upper Intermediate
        \end{cvSkillItem}
        \begin{cvSkillItem}{Französisch}{Grundkenntnisse}
          Beginner
        \end{cvSkillItem}
      \end{multicols}

    \cvSubsection{Programmiersprachen}
      \vspace{-1.8em}
      \begin{multicols}{2}
        \begin{cvSkillItem}{C/C++}{Fortgeschrittene Kenntnisse}
          % Erfahrung in den folgenden Bereichen:
          \begin{cvItemize}
            \item Standards: C++98, C++11, C++14, C++17
            \item Bibliotheken: Boost, Doctest, Qt, SFML, OpenGL
            \item Concurrency: Threads, OpenMP, MPI, CUDA, SSE, AVX
            \item Compiler: GCC, Clang, Intel C++ Compiler
            \item Build Systeme: CMake, Make, qmake, Meson, build2
            \item Support: Git, Valgrind, clang-tidy, clang-format
          \end{cvItemize}
        \end{cvSkillItem}
        \begin{cvSkillItem}{CMake}{Fortgeschrittene Kenntnisse}
          Verwendung eines konsistenten und modernen Standards
        \end{cvSkillItem}
        \begin{cvSkillItem}{Python}{Grundlegende Kenntnisse}
          Erfahrung beim  Verstehen und Interagieren mit \\ existierendem Code
        \end{cvSkillItem}
        \begin{cvSkillItem}{Java}{Grundlegende Kenntnisse}
        \end{cvSkillItem}
        \begin{cvSkillItem}{LaTeX}{Fortgeschrittene Kenntnisse}
          \begin{cvItemize}
            \item Entwicklung eigener Pakete
            \item Anpassung externer Pakete
          \end{cvItemize}
        \end{cvSkillItem}
      \end{multicols}

    \cvSubsection{Betriebssysteme, DevOps, Webdesign und Weiteres}
      \vspace{-1.8em}
      \begin{multicols}{2}
        \begin{cvSkillItem}{Linux}{Fortgeschrittene Kenntnisse}
          Verwendung von Arch Linux und Ubuntu
        \end{cvSkillItem}
        \begin{cvSkillItem}{Windows}{Grundlegende Kenntnisse}
          Erfahrung mit Windows 7 und Windows 10
        \end{cvSkillItem}
        \begin{cvSkillItem}{Docker}{Fortgeschrittene Kenntnisse}
          % Erstellung eigener Images und Container auf DockerHub
        \end{cvSkillItem}
        \begin{cvSkillItem}{CircleCI}{Fortgeschrittene Kenntnisse}
        \end{cvSkillItem}
        \begin{cvSkillItem}{Codecov}{Grundlegende Kenntnisse}
        \end{cvSkillItem}
        \begin{cvSkillItem}{Jekyll}{Grundlegende Kenntnisse}
          Erstellung von Projektdokumentation für GitHub Pages
        \end{cvSkillItem}
        \begin{cvSkillItem}{HTML}{Grundlegende Kenntnisse}
        \end{cvSkillItem}
        \begin{cvSkillItem}{Gnuplot}{Fortgeschrittene Kenntnisse}
        \end{cvSkillItem}
        \begin{cvSkillItem}{Blender}{Fortgeschrittene Kenntnisse}
        \end{cvSkillItem}
        \begin{cvSkillItem}{Geogebra}{Grundlegende Kenntnisse}
        \end{cvSkillItem}
      \end{multicols}

  \cvSection{Praxiserfahrung}
    \cvSubsection{Fraunhofer ITWM Kaiserslautern: Competence Center High Performance Computing (CC HPC)}
      \begin{cvTimeItem}{Sep.~2012}{Praktikum}
        Implementierung einer Raytracing-Engine beschleunigt durch LBVH basierend auf dem Morton-Code
      \end{cvTimeItem}
      \begin{cvTimeItem}{Okt.~2013 - Jun.~2017}{Wissenschaftliche Hilfskraft}
        \begin{cvItemize}
          \item Fachkenntnis und Erfahrung in den Bereichen \cvQuote{Programmoptimierung in C++ und C}, \cvQuote{Compilerbau}, \cvQuote{Computerhardware}, \cvQuote{Parallel Computing} und \cvQuote{Computergrafik}
          \item Implementierung von echtzeitfähigen Raytracing-Algorithmen auf der CPU und GPU unter Verwendung von State-of-the-Art-Verfahren und professioneller Werkzeuge, wie OpenGL, Qt und CUDA
          \item Unterstützung bei der Entwicklung eines statistisch-basierten Analysewerkzeuges für seismische Daten durch Implementierung von Histogrammen, Kerndichteschätzern und Farbtabellen
          \item Implementierung von Schnittstellen zur Verarbeitung des \cvQuote{Wavefront OBJ} Dateiformates
          \item Aufbereitung und Nachbearbeitung diverser Szenenmodelle mithilfe von Blender
        \end{cvItemize}
      \end{cvTimeItem}

    \cvSubsection{Friedrich-Schiller-Universität Jena}
      \begin{cvTimeItem}{Okt.~2017 - Apr.~2018}{Wissenschaftliche Hilfskraft}
        \begin{cvItemize}
          \item Übungsleiter im Fach \cvQuote{Mathematische Methoden der Physik I}
          \item Erstellen der Aufgabenzettel und Musterlösungen mithilfe von Latex
          \item Entwicklung einer sich automatisch kompilierenden Latex-basierten Aufgabendatenbank
        \end{cvItemize}
      \end{cvTimeItem}
      \begin{cvTimeItem}{Sep.~2018}{Tutor}
        \begin{cvItemize}
          \item Einführungskurs von über zehn Tagen in die Programmiersprache C++ auf der Basis moderner Standards und Werkzeuge
          \item Einführungskurs von über fünf Tagen in die Erstellung wissenschaftlicher Arbeiten mithilfe von LaTeX
        \end{cvItemize}
      \end{cvTimeItem}

  \cvSection{Weitere Interessen und Aktivitäten}
    \begin{multicols}{2}
      \begin{cvTimeItem}{seit 2008}{Gitarre, E-Gitarre}
        \begin{cvItemize}
          \item Unterricht zwei- bis viermal im Monat
          \item Diverse Soloauftritte
        \end{cvItemize}
      \end{cvTimeItem}
      \begin{cvTimeItem}{Jan.~2014 - Dez.~2018}{Band Headedge: Lead-Gitarre}
        \begin{cvItemize}
          \item Monatliche Auftritte innerhalb und außerhalb von Jena
          \item Sieger des Jenaer Nachwuchsbandwettbewerbes
          \item Veröffentlichung eines eigenen Studioalbums
        \end{cvItemize}
      \end{cvTimeItem}
      \begin{cvTimeItem}{Jan.~2016 - Okt.~2018}{Tanzclub Kristall Jena e.V.: Latein-Tuniertanz}
      \end{cvTimeItem}
      \begin{cvTimeItem}{seit Okt.~2018}{Wöchentliches C++-Meeting: Leiter}
      \end{cvTimeItem}
      \begin{cvTimeItem}{seit Mär.~2019}{Unisport: Kung Fu, Akrobatik}
      \end{cvTimeItem}
    \end{multicols}
\end{document}