\documentclass[a4paper,8pt]{cv}

\newcommand{\cvFirstName}{Markus}
\newcommand{\cvLastName}{Pawellek}
\newcommand{\cvAddress}{Arvid-Harnack-Straße 12 \\ 07743 Jena \\ Deutschland}
\newcommand{\cvBirthday}{7.~Mai~1995}
\newcommand{\cvBirthplace}{Meiningen,~Deutschland}
\newcommand{\cvMobile}{+49~173~7262913}
\newcommand{\cvMail}{markuspawellek@gmail.com}
\newcommand{\cvGitHub}{lyrahgames}
\newcommand{\cvPhoto}{photo.png}

\begin{document}
  \cvHead
  \cvSection{Ausbildung}
    \cvSubsection{Goetheschule Ilmenau Staatliches Gymnasium}
      \begin{cvTimeItem}{Sep.~2009 - Jun.~2013}{Allgemeine Hochschulreife (1,2)}
        \begin{cvItemize}
          \item Besuch der mathematisch-naturwissenschaftlichen Spezialklassen
          \item Abschluss mit sehr gutem Erfolg in den erweiterten Fächern Mathematik, Physik und Informatik
          \item Abgabe zweier Facharbeiten in den Bereichen \cvQuote{Compilerbau} und \cvQuote{Raytracing}
          \item Preisträger mehrerer Mathematik- und Physikolympiaden
          \item Dreijähriger Besuch der Elektronik-AG
        \end{cvItemize}
      \end{cvTimeItem}

    \cvSubsection{Technische Universität Ilmenau}
      \begin{cvTimeItem}{Okt. 2011 - Sep. 2012}{Frühstudium: Experimentalphysik (1,0)}
      \end{cvTimeItem}

    \cvSubsection{Friedrich-Schiller-Universität Jena}
      \begin{cvTimeItem}{Okt. 2013 - Sep. 2017}{B.Sc. Physik (1,7)}
        Abschlussarbeit \cvQuote{Generierung von Irradiance Maps} (1,3) über das Cachen der diffusen Lichtverteilung einer Szene, um deren Darstellung in Echtzeit mithilfe des Raytracing-Algorithmus zu ermöglichen
      \end{cvTimeItem}
      \begin{cvTimeItem}{Okt. 2015 - Sep. 2018}{B.Sc. Mathematik (1,4)}
        \begin{cvItemize}
          \item Spezialisierung auf den Bereich \cvQuote{Theoretische Informatik}
          \item Abschlussarbeit \cvQuote{Implementierung einer Finite-Elemente-Methode auf der Grafikkarte} (1,0) über die numerische Simulation von Lösungen der idealen Wellengleichung auf zweidimensionalen Untermannigfaltigkeiten
        \end{cvItemize}
      \end{cvTimeItem}
      \begin{cvTimeItem}{seit Okt.~2017}{M.Sc. Physik}
        \begin{cvItemize}
          \item Spezialisierung auf den Bereich \cvQuote{Quanten- und Gravitationstheorie} mit Nebenfach \cvQuote{Astronomie}
          \item Abschlussarbeit \cvQuote{Design and Implementation of Pseudo Random Number Generators for Simulation in Physics}
        \end{cvItemize}
      \end{cvTimeItem}

  \cvSection{Fähigkeiten}
    \cvSubsection{Sprachen}
      \vspace{-1.8em}
      \begin{multicols}{2}
        \begin{cvSkillItem}{Deutsch}{Muttersprache}
        \end{cvSkillItem}
        \begin{cvSkillItem}{Englisch}{Fließend in Wort und Schrift}
          Upper Intermediate
        \end{cvSkillItem}
        \begin{cvSkillItem}{Französisch}{Grundkenntnisse}
          Beginner
        \end{cvSkillItem}
      \end{multicols}

    \cvSubsection{Programmiersprachen}
      \vspace{-1.8em}
      \begin{multicols}{2}
        \begin{cvSkillItem}{C/C++}{Fortgeschritten}
          Advanced \hfill 9 Jahre Erfahrung \\[1pt]
          Verwendung in allen Softwareprojekten \\
          (z.B. N-Körper-Problem, Fluidsimulationen)
          \begin{cvItemize}
            % \item Projekte: N-Körper-Problem, Strömungssimulationen
            \item Standards: C++98, C++11, C++14, C++17
            \item Bibliotheken: Boost, Doctest, Qt, SFML, OpenGL
            \item Concurrency: Threads, OpenMP, MPI, CUDA, SSE, AVX
            \item Compiler: GCC, Clang, Intel C++ Compiler
            \item Build Systeme: CMake, Make, qmake, Meson, build2
            \item Support: Git, Valgrind, clang-tidy, clang-format
          \end{cvItemize}
        \end{cvSkillItem}
        \begin{cvSkillItem}{Python}{Einsteiger}
          Novice\hfill 2 Jahre Erfahrung
          % Erfahrung beim  Verstehen und Interagieren mit \\ existierendem Code
        \end{cvSkillItem}
        \begin{cvSkillItem}{CMake}{Fortgeschritten}
          Advanced\hfill 2 Jahre Erfahrung \\[1pt]
          Verwendung in C/C++-Projekten zum Kompilieren, Testen und Installieren mit konsistentem und modernem Standard
        \end{cvSkillItem}
        \begin{cvSkillItem}{Java}{Grundkenntnis}
          Beginner
        \end{cvSkillItem}
        \begin{cvSkillItem}{LaTeX}{Fortgeschritten}
          Advanced\hfill 8 Jahre Erfahrung\\[1pt]
          Verwendung für alle Ausarbeitungen und Präsentationen
          \begin{cvItemize}
            \item Entwicklung eigener Pakete und Klassen
            \item Anpassung externer Pakete
          \end{cvItemize}
        \end{cvSkillItem}
      \end{multicols}

    \cvSubsection{Betriebssysteme, DevOps, Webdesign und Weiteres}
      \vspace{-1.8em}
      \begin{multicols}{2}
        \begin{cvSkillItem}{Linux}{Fortgeschritten}
          Advanced\hfill 8 Jahre Erfahrung \\[1pt]
          Verwendung von Arch Linux und Ubuntu im Alltag und für die Bearbeitung von Projekten
        \end{cvSkillItem}
        \begin{cvSkillItem}{Windows}{Fortgeschrittener Einsteiger}
          Intermediate
        \end{cvSkillItem}
        \begin{cvSkillItem}{Docker}{Fortgeschrittener Einsteiger}
          Intermediate \hfill 1 Jahr Erfahrung \\[1pt]
          Erstellung eigener Images für CI-Umgebungen
        \end{cvSkillItem}
        \begin{cvSkillItem}{CircleCI}{Fortgeschrittener Einsteiger}
          Intermediate \hfill 1 Jahr Erfahrung
          \begin{cvItemize}
            \item Automatisches Testen von Code
            \item Automatische Generierung von Dokumentationen
          \end{cvItemize}
        \end{cvSkillItem}
        \begin{cvSkillItem}{Codecov}{Einsteiger}
          Novice
        \end{cvSkillItem}
        \begin{cvSkillItem}{Jekyll}{Einsteiger}
          Novice \\[1pt]
          Erstellung von Projektdokumentation für GitHub Pages
        \end{cvSkillItem}
        \begin{cvSkillItem}{HTML}{Fortgeschrittener Einsteiger}
          Intermediate \hfill 4 Jahre Erfahrung \\[1pt]
          Erstellung eigener Homepages
        \end{cvSkillItem}
        \begin{cvSkillItem}{Gnuplot}{Fortgeschrittener Einsteiger}
          Intermediate \hfill 6 Jahre Erfahrung \\[1pt]
          Verwendung im Studium und in Projekten für die Erzeugung von Plots als Vektorgrafiken in LaTeX-Dokumenten
        \end{cvSkillItem}
        % \begin{cvSkillItem}{Blender}{Fortgeschrittene Kenntnisse}
        % \end{cvSkillItem}
        % \begin{cvSkillItem}{Geogebra}{Grundlegende Kenntnisse}
        % \end{cvSkillItem}
      \end{multicols}

  \cvSection{Praxiserfahrung}
    \cvSubsection{Fraunhofer ITWM Kaiserslautern: Competence Center High Performance Computing (CC HPC)}
      \begin{cvTimeItem}{Sep.~2012}{Praktikum}
        Implementierung einer Raytracing-Engine beschleunigt durch LBVH basierend auf dem Morton-Code
      \end{cvTimeItem}
      \begin{cvTimeItem}{Okt.~2013 - Jun.~2017}{Wissenschaftliche Hilfskraft}
        \begin{cvItemize}
          \item Erhalt von Fachkenntnis und Erfahrung in den Bereichen \cvQuote{Programmoptimierung in C++ und C}, \cvQuote{Compilerbau}, \cvQuote{Computerhardware}, \cvQuote{Parallel Computing} und \cvQuote{Computergrafik}
          \item Implementierung von echtzeitfähigen Raytracing-Algorithmen auf der CPU und GPU unter Verwendung von State-of-the-Art-Verfahren und professioneller Werkzeuge, wie OpenGL, Qt und CUDA
          \item Unterstützung bei der Entwicklung eines statistisch-basierten Analysewerkzeuges für seismische Daten durch Implementierung von Histogrammen, Kerndichteschätzern und Farbtabellen
          \item Implementierung von Schnittstellen zur Verarbeitung des \cvQuote{Wavefront OBJ} Dateiformates
          \item Aufbereitung und Nachbearbeitung diverser Szenenmodelle mithilfe von Blender
        \end{cvItemize}
      \end{cvTimeItem}

    \cvSubsection{Friedrich-Schiller-Universität Jena}
      \begin{cvTimeItem}{Okt.~2017 - Apr.~2018}{Wissenschaftliche Hilfskraft}
        \begin{cvItemize}
          \item Übungsleiter im Fach \cvQuote{Mathematische Methoden der Physik I}
          \item Erstellen der Aufgabenzettel und Musterlösungen mithilfe von LaTeX
          \item Entwicklung einer sich automatisch kompilierenden LaTeX-basierten Aufgabendatenbank
        \end{cvItemize}
      \end{cvTimeItem}
      \begin{cvTimeItem}{Sep.~2018}{Tutor}
        \begin{cvItemize}
          \item Einführungskurs in die Programmiersprache C++ auf der Basis moderner Standards und Werkzeuge
          \item Einführungskurs in die Erstellung wissenschaftlicher Arbeiten mithilfe von LaTeX
        \end{cvItemize}
      \end{cvTimeItem}

  \cvSection{Weitere Interessen und Aktivitäten}
    \vspace{-1em}
    \begin{multicols}{2}
      \begin{cvTimeItem}{seit 2008}{Gitarre, E-Gitarre}
        \begin{cvItemize}
          \item Unterricht zwei- bis viermal im Monat
          \item Diverse Soloauftritte
        \end{cvItemize}
      \end{cvTimeItem}
      \begin{cvTimeItem}{Jan.~2014 - Dez.~2018}{Band Headedge: Lead-Gitarre}
        \begin{cvItemize}
          \item Musikrichtung: Eclectic Rock
          \item Monatliche Auftritte innerhalb und außerhalb von Jena
          \item Sieger des Jenaer Nachwuchsbandwettbewerbes
          \item Veröffentlichung eines eigenen Studioalbums
        \end{cvItemize}
      \end{cvTimeItem}
      \begin{cvTimeItem}{Jan.~2016 - Okt.~2018}{Latein-Tuniertanz}
      \end{cvTimeItem}
      \begin{cvTimeItem}{seit Okt.~2018}{Wöchentliches C++-Meeting: Leiter}
      \end{cvTimeItem}
      \begin{cvTimeItem}{seit Mrz.~2019}{Unisport: Kung Fu, Akrobatik}
      \end{cvTimeItem}
    \end{multicols}
\end{document}